\chapter{Introduction}
% Introductory stuff
\section{The Foundation of Evolutionary Biology}
Humans have long sought to understand biological variation. To this end, ancient scholars proposed numerous theories of heredity, including that of the \textit{inheritance of acquired characteristics}. It is clear that members of the same family tend to resemble one another. Similarly, members of the same community, or population, tend to share physical characteristics that may be distinct from members of other populations. It is conceivable that children are predetermined to share traits with their parents. Perhaps there exists some quality of living things that allows traits to be recorded throughout one's life and passed down to future generations. Variations in physical characteristics between populations would then be a result of differing records of acquired traits.   

In \textit{On Airs, Waters, and Places}, Hippocrates writes about a community of people known as "Macrocephali" or "Longheads" who would mechanically lengthen the skulls of their children. Hippocrates posits that generations of this practice influenced the natural state of this community such that children were eventually born with long heads without undergoing physical lengthening. 
\begin{quote}
    There is no other race of men which have heads in the least resembling theirs. At first, usage was the principal cause of the length of their head, but now nature cooperates with usage. They think those the most noble who have the longest heads. It is thus with regard to the usage: immediately after the child is born, and while its head is still tender, they fashion it with their hands, and constrain it to assume a lengthened shape by applying bandages and other suitable contrivances whereby the spherical form of the head is destroyed, and it is made to increase in length. Thus, at first, usage operated, so that this constitution was the result of force: but, in the course of time, it was formed naturally; so that usage had nothing to do with it. \cite{hippocrates_airs_waters_places}
\end{quote} 

The inheritance of acquired characteristics remained the predominant explanation of biological variation for more than 2,000 years after Hippocrates. Enlightenment thinkers contributed closely aligned theories with some additional insights, but considerable added terminology. Chief among these are \textit{Lamarckian inheritance} and \textit{Darwinian pangenesis}. Jean-Baptiste Lamarck is famously credited with having developed an evolutionary framework based on the inheritance of acquired characteristics, or "soft inheritance", as he called it.  \cite{lamarck_1914} Lamarck proposed that living things develop in increasing complexity as they adapt to the circumstances of their environment. Novel physical traits arise from their utility to the organism and are subsequently passed on to offspring. Traits that are not of use to the organism are lost. Many young biologists learn of Lamarck's illustration of giraffes stretching and slowly lengthening their necks to reach leaves on trees, leading to subsequent generations with naturally longer necks. This example is often presented as ridiculous when shown in opposition to the \textit{correct} frameworks of Charles Darwin and Alfred Russel Wallace. The theory of evolution by natural selection proposed in \textit{On the Origin of Species by Means of Natural Selection, of the Preservation of Favoured Races in the Struggle for Life} is a fundamental component of modern evolutionary biology. \cite{darwin_1859} Still, Darwin's explanation for heredity was remarkably similar to that of Lamarck and even that of Hippocrates. Darwin's "pangenesis" hypothesis claimed that each part of the body emitted particles of heritable information called "gemmules".\cite{darwin_1868}  If a part of the body was altered in response to one's environment, this part would create altered gemmules. \cite{darwin_pangenesis_1871} These particles were said to aggregate in the gonads to be passed on to offspring.  This hypothesis was invalidated following the rediscovery of Mendelian inheritance in 1900, but the Lamarck-Darwin dichotomy remains a critical teaching construct in biology textbooks. \cite{holterhoff_history_2014} \cite{zirkle_inheritance_1935}


Johann Mendel grew up on a family farm in the village of Hynčice, located in the modern-day Czech Republic. He worked as a gardener during his childhood before obtaining an education in philosophy and physics, but struggled to afford his schooling. Mendel found financial security in joining the Augustinian friars as a monk, adopting the first name "Gregor".  In 1856, Gregor Mendel began studying variation in pea plants after receiving inspiration from his professor Johann Karl Nestler, who studied sheep variation.\cite{henig_2000} He focused on the inheritance of seven discrete characteristics in the common pea plant \textit{Pisum sativum}: seed texture, seed color, pod texture, pod color, flower color, flower position, and plant height. Possible trait variants were indicated by combinations of capital and lower case letters (e.g. \textbf{\textit{Aa}} represented a generation produced by pollination from some round \textbf{\textit{A}} and some wrinkled seeds \textbf{\textit{a}}). The counts of trait appearances across multiple successive generations were recorded, while preventing pollination from foreign plants. Mendel identified several significant principles, coining "recessive" in reference to variants that were observed in preceding and subsequent generations, but masked by other "dominant" variants in some generations. This concept is now referred to as the \textit{Law of Dominance}. Mendel demonstrated that egg and pollen cells carried one of each variant (now called the \textit{Law of Segregation}), and these variants separate independently into egg and pollen cells (\textit{Law of Independent Assortment}). These are all together now known as \textit{Mendel's laws of inheritance}. Mendel's findings were published in the now famous "Experiments on Plant Hybridization" \cite{mendel_1865} in 1866, but were largely ignored until 1900, when his ideas were rediscovered and brought before an incredulous scientific community. The ensuing debate in search of the \textit{correct} mechanisms for biological variation would give rise to modern population genetics. \cite{bowler_2003}

\subsection{The Mendelian Paradox (possible side discussion)}

%% Maybe include discussion about Mendelian paradox?
%% \cite{fisher_1936}

\newpage
\section{The Modern Synthesis}
Bateson and the "Mendelians" vs. Pearson and the "Biometricians"\cite{bateson_1902}


Fisher's 1918 paper and \textit{The Genetical Theory of Natural Selection} -- reconciling Mendelian genetics with evolution by natural selection


Julian Huxley's \textit{Evolution: The Modern Synthesis}


\section{Natural Selection and Neutral Theory}
Return to Darwin's theory of evolution by natural selection and "survival of the fittest"  


Sewall Wright and genetic drift & Fisher argued that drift played a minor role in evolution, with selection as the main driver


Kimura neutral theory and Infinite Sites Model


Reconciling the "neutralist vs. selectionist debate"
\\
Kern Hanhn 2018 "Neutral Theory in Light of Natural Selection" -> selection \textbf{is} the driving force of evolution

\section{Spatial Modeling in Population Genetics}
Isolation by distance


Novembre et. al.2008


The Kimura Stepping Stone model  


Novembre and Stephens 2008 human genetic data shows continuous spatial structure 


Allison Etheridge on Spatial modeling in popgen 

\section{GWAS and Sampling Processes}
The genomics revolution & GWAS


Bias in genomic data collection


The Site Frequency Spectrum as a quantitative summary of segregating sites 


Sampling process distorts important quantities in popgen analyses --- SFS

McVean 2009, Battey 2019 

\newpage
\section{Project Rationale and Significance}
%% Need to reformat citations!
Quantitative and population geneticists are interested in finding statistical associations between genetic variants and particular traits. Genome-wide association studies (GWAS) are one common method for detecting associations and are especially useful when attempting to predict an individual’s risk for disease. Disease GWAS function by comparing a large set of genotypes to the known disease phenotypes of members in the group. If members with a particular allele, or variant of a gene, are shown to have a significantly higher probability of also being in the group with the disease, then the GWAS supports the hypothesis that the allele increases one’s risk of developing the disease. GWAS typically estimate the genetic effects on a phenotype by conducting a linear regression as follows:\cite{dudbridge_2013}

\begin{equation}
    \hat{\bf{Y}} = \boldsymbol{\hat{\beta}} \bf{G} + \boldsymbol{\epsilon} = \sum_{i=1}^{n} (\hat{\beta_i} G_i + \epsilon_i) 
\end{equation}

Where $\hat{\bf{Y}}$ represents the phenotype expressed as a linear combination of $n$ genetic loci in matrix $\bf{G}$ scaled by their effect sizes $\hat{\beta}$. Error terms are accounted for in $\boldsymbol{\epsilon}$ and are independent from $\bf{G}$. It is important to note that $\hat{\beta}$ is an estimated quantity. Researchers have proposed a number of statistical methods for estimating effect sizes. \cite{Meuwissen_et.al._2001}



Conducting a GWAS requires a large set of genomic data which is gathered by a sampling process. As a result, association studies feature significant geographic sampling bias. For example, the UK BioBank contains samples of people who have migrated to the United Kingdom from around the world, but still represents a small fraction of global diversity [?]. Therefore, it is important to assess if GWAS results from geographically localized studies, such as those from the UK BioBank, can predict disease risk in other geographic regions. This project adds to the existing literature of theoretical models for studying deleterious alleles [8][9][10], but is one of the first to consider the influence of spatial sampling bias. Then, the goal of this project is to create a framework for evaluating how geographic sampling bias may affect the detection of rare deleterious alleles. Towards these ends, we note the following aims:	

\begin{enumerate}
    \item Develop theoretical models to derive the distribution of allele frequencies as a function of the spatial sampling scheme.
    \item Determine the effect of sampling schemes on allele detection probability.
\end{enumerate}


Preliminary studies have shown that deleterious, disease-causing alleles are often rare, and rare alleles are often geographically localized. When conducting a GWAS, a geographically explicit sampling scheme must be applied. However, there does not exist a theoretical framework for understanding the interaction of strong natural selection and biased spatial sampling. Such a framework would potentially improve the power of a GWAS to detect deleterious alleles
The geographic distribution of rare deleterious alleles has not been rigorously studied. Previous work in the Novembre lab has visualized the spatial localization of rare alleles in global data sets [?], but deeper considerations have not been given to the effect of biased sampling. This project will formalize the intuition circulating discussion of the sampling of rare deleterious alleles. We will build novel derivations from well-known population genetics models to improve the theoretical framework for studying rare alleles. 


This work is ultimately interested in understanding how the process of sampling may affect the power of a GWAS to detect deleterious alleles, as this interaction has not been previously studied. This theoretical framework aims to inform the interpretation of GWAS results and quantify the effect of sampling. The project aims to relate the power to detect disease-causing alleles to the evolutionary forces acting upon them and the constructed sampling scheme. It is predicted that certain sampling schemes may increase the power of a GWAS for certain alleles. The model proposed allows one to quantitatively determine the optimal sampling strategies for given deleterious alleles of varying rarity. This model can ultimately be used to inform future GWAS study design by accounting for sampling effects in data collection, possibly increasing the power of a GWAS to detect rare alleles. This could likely improve GWAS predictions of individual disease risk.


\begin{center}
    \textbf{Primary Question:} \\
    How  does geographic sampling bias affect the detection of rare deleterious alleles?
\end{center}