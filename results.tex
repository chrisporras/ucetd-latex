\chapter{Results}
\section{Important Quantities from Our Model}
As shown in the previous chapter, our model describes the change in allele frequencies over time and space. Evolutionary forces act to shape the distribution of alleles across our simulated geographic habitat. There are two immediate questions that we will utilize our model to answer:

\begin{enumerate}
    \item How much of the allele do we expect to observe?
    \item Where can we expect to find the allele?
\end{enumerate}


For a more intuitive understanding of what each of these questions represent, let us examine the raw simulation output from our model. Allele frequencies follow stochastic trajectories illustrated in examples below:

\begin{figure}[h]
    \centering
    \includegraphics{}
    \caption{Caption}
    \label{fig:time_series}
\end{figure}


\begin{figure}[h]
    \centering
    \includegraphics{}
    \caption{Caption}
    \label{fig:space_series}
\end{figure}


\subsection{The average allele frequency}
 

\subsection{The geographic range of the allele}


\section{Computing the Expected Site Frequency Spectrum}

\section{Measuring Missing Heritability}