\abstract
% Enter Abstract here
Genome-wide association studies (GWAS) have shown that much of the variation in disease risk is due to rare deleterious alleles. When considering the geographic distribution of a population, studies have shown these alleles are removed by selection before they can spread beyond their original location. However, the geographic distributions of these alleles have not been characterized in detail. This study aims to develop theoretical models to understand how natural selection, geographic structure, and geographic sampling bias interact to determine the inferred local and global genetic architecture of a trait. Our theoretical framework aims to inform the interpretation of GWAS results and quantify the effect of sampling.

\begin{center}
    \textbf{Primary Question:} \\
    How  does geographic sampling bias affect the detection of rare deleterious alleles?
\end{center}


Our primary question can be broken down into the following components:

\begin{enumerate}
    \item Natural selection acts on alleles to determine the traits of a population
    \item Geography influences the distribution of genetic variation 
    \item GWAS identify associated alleles by sampling individuals from particular geographic regions
\end{enumerate}

This dissertation examines these components first from a broad, historical perspective and later through the framework constructed in our theoretical model. The main contribution of this work is an incorporation of the geographic sampling process into population genetic models. We relate commonly studied population parameters such as the migration rate $m$ and the selection coefficient $s$ to the observation of rare deleterious alleles from population samples.     