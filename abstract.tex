\abstract
% Enter Abstract here
Genome-wide association studies (GWAS) have shown that much of the variation in disease risk is due to rare deleterious alleles. When considering the geographic distribution of a population, studies have shown these alleles are removed by selection before they can spread beyond their original location. However, the spatial distributions of these alleles have not been characterized in detail. This study aims to develop theoretical models to understand how natural selection, spatial structure, and geographic sampling bias interact to determine the inferred local and global genetic architecture of a trait. Our theoretical framework aims to inform the interpretation of GWAS results and quantify the effect of sampling.