\chapter{Discussion}
Genetic variation is structured by geography. In human populations, genetic variation spans continuously across the globe \cite{novembre_interpreting_2008}\cite{1k_genomes}. There exist alleles unique to populations from certain geographic regions and these alleles tend to be rare \cite{slatkin_isolation_1993}\cite{marcus_visualizing_2017}. To detect associations between phenotypes and rare alleles, it makes sense that individuals chosen to participate in such an association study be from the geographic region where this allele exists. This applies even more so to genome-wide association studies of disease. Evolutionary biology predicts that disease-associated alleles with a large effect on conveying the trait would be removed from a population by natural selection. Therefore, these deleterious alleles would be kept at low frequencies in the population and be even more likely to exist in localized geographic areas. Of course, researchers don't know the underlying geographic distribution of alleles. To identify certain alleles within a population, researchers must sample individuals with the trait of interest, often sampling from a specific region \cite{bycroft_uk_2018} \cite{1k_genomes}. This imposes a geographic \textit{sampling bias}.


Our work is interested in understanding the interaction between the underlying population structure and the GWAS sampling process \ref{fig:schematic}. How does where researchers sample determine what they are able to observe? A theoretical framework that incorporates these interactions has yet to be developed. To accomplish this, we construct and analyze theoretical models. We expand the classic Wright-Fisher \cite{wright_evolution_1931} \cite{fisher_mathematical_1922} and Stepping Stone models \cite{kimura_stepping_1964} to incorporate the evolutionary forces of selection, mutation, migration, and genetic drift (equation: \ref{eq:model}). From this, we simulate the geographic distribution of rare alleles under purifying selection. We sample from this distribution using Gaussian sampling kernels of varying width, to simulate narrow versus broad geographic sampling. 


We find that as the sampling width increases, the probability of detecting localized alleles decreases exponentially (figure: \ref{fig:sfs}). However the probability of observing many different alleles increases (figure: \ref{fig:sampling_curves}), but the probability of observing more than one copy of each allele is diminished. This quantitatively illustrates a trade-off in GWAS sampling. Sampling broadly improves the chance of seeing more associated alleles, but with a lower number of copies and likely weaker power to detect strong associations. Sampling narrowly improves the probability of detecting multiple copies, but as seen in figure: \ref{fig:missing_h}, the fraction of heritability explained in such a sample is worse than sampling broadly if the sample isn't close enough to the focal population where the allele is at highest frequency. We provide a quantitative framework to understand the effects of sampling bias on GWAS. 


The site frequency spectrum (SFS) is an observed quantity summarizing the distribution of allele frequencies in a population \cite{lapierre_accuracy_2017}. We derive the expected site frequency spectrum with the influence of sampling \ref{fig:sfs}. Ultimately, we aim to relate this form of the expected SFS along with our calculation of the missing heritability to our population parameters, particularly ($s$, $w$, for selection and sampling). However, we demonstrate preliminary findings, noting these results in future directions. 


Our model relies on many of the Wright-Fisher assumptions. Namely, we assume a fixed population size, random mating, non-overlapping generation times, and symmetric mutation. Our model is generally flexible enough to incorporate some more complex population dynamics. For instance, modifying our model to include asymmetric mutation would be as simple as substituting ($\mu(1-2f) = \mu_{a}(1-f) - \mu_{A}f$). Because we are most interested in selection, migration, and sampling, many other processes can be simplified for our initial work and expanded in future directions. We also model haploid individuals and share the migratory assumptions of the Stepping Stone model. 


Future directions include not just deriving summary statistics in terms of population parameters, or adding parameters to our model, but also confirming our theoretical results with real data. It is clear that sampling processes distort observed population structure in simulated data \cite{mcvean_genealogical_2009} \cite{battey_space_2019}, but researchers have yet to explore these interactions "\textit{in vivo}".  


